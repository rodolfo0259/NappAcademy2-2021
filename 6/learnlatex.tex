\documentclass{article}

\usepackage{graphicx}

\graphicspath{ {./prints} }
\usepackage[section]{placeins}
\usepackage[T1]{fontenc}

\author{Rodolfo Contieri Baldin}
\title{NappAcademy Sprint6 Exercicios}
\date{}

\begin{document}
\maketitle


\newpage
\section{Protocolo FTP}


\begin{figure}[ht]
    \centering
    \includegraphics[width=.9\linewidth,height=.8\textheight,keepaspectratio]{print1-1_Concetado.png}
    \caption{Print 1-1}
\end{figure}


\begin{figure}[ht]
    \centering
    \includegraphics[width=.9\linewidth,height=.8\textheight,keepaspectratio]{print3-1-1_help(ftp)_methods.png}
    \caption{Print 3-1-1 FTP methods}
\end{figure}

\begin{figure}[ht]
    \centering
    \includegraphics[width=.9\linewidth,height=.8\textheight,keepaspectratio]{print3-1-1_README_nao_gerado_ftp_parte1.png}
    \caption{Print 3-1-1 executando parte1/ftp.py}
\end{figure}

\begin{figure}[ht]
    \centering
    \includegraphics[width=.9\linewidth,height=.8\textheight,keepaspectratio]{print3-1-1_README_gerado_ftp_Pratica.png}
    \caption{Print 3-1-1 executando Prática/ftp.py}
\end{figure}

\newpage
\section{Consumo API}

\begin{figure}[ht]
    \centering
    \includegraphics[width=.9\linewidth,height=.8\textheight,keepaspectratio]{print4-1_json.png}
    \caption{Print 4-1 response in JSON}
\end{figure}

\begin{figure}[ht]
    \centering
    \includegraphics[width=.9\linewidth,height=.8\textheight,keepaspectratio]{print4-2_xmldata.png}
    \caption{Print 4-2 response in XML}
\end{figure}

\begin{figure}[ht]
    \centering
    \includegraphics[width=.9\linewidth,height=.8\textheight,keepaspectratio]{print4-3_piped.png}
    \caption{Print 4-3 response in PIPED}
\end{figure}


\begin{figure}[ht]
    \centering
    \includegraphics[width=.9\linewidth,height=.8\textheight,keepaspectratio]{print5-1_Avenida.png}
    \caption{Print 5-1 Descobrir se é uma avenida}
\end{figure}

\begin{figure}[ht]
    \centering
    \includegraphics[width=.9\linewidth,height=.8\textheight,keepaspectratio]{print7-1_SPdados20210820.png}
    \caption{Print 7-1 dados de SP em 2021-08-20}
\end{figure}

\begin{figure}[ht]
    \centering
    \includegraphics[width=.9\linewidth,height=.8\textheight,keepaspectratio]{print8-1_paises_casos.png}
    \caption{Print 8-1 casos em todos os paises}
\end{figure}

\newpage
\section{Raspagem de Dados}

\begin{figure}[ht]
    \centering
    \includegraphics[width=.9\linewidth,height=.7\textheight,keepaspectratio]{print9-1_link_status_code.png}
    \caption{Print 9-1 request status de cada link}
\end{figure}


\newpage
\section{Perguntas}

\noindent
\textbf{Pergunta 2.}\par
Ocorreu comunicacao ?\newline

-- Sim, durante a conexão e desconexão:

"Concetado por ('127.0.0.1', 59540)"

"Finalizando conexao do cliente ('127.0.0.1', 59540)"
\newline

\noindent
\textbf{Pergunta 6.1.}\par
Qual o tipo de dado de dados{\_}SP ?\newline

print(type(dados{\_}SP)) = <class 'dict'>
\newline

\noindent
\textbf{ Pergunta 10.}\par
Um dos status\_code apresenta número 999. Qual é esta URL ? Pesquise na internet e explique porque isso Ocorreu \newline

(999, 'https://www.linkedin.com/company/nappsolutionsbr/')\par

O codigo http 999 == request denied\par
A url do Linkedin não aceita esse tipo de requests, é necessario o uso da REST API especifica deles: python3-linkedin\par

\end{document}